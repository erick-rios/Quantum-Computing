\documentclass{article}
\usepackage[utf8]{inputenc}
\usepackage[spanish]{babel}
\usepackage{graphicx} % Required for inserting images
\usepackage{geometry}
\usepackage{pgfplots}
\usepackage{float}
\usepackage{enumitem}
\usepackage{titlesec}
\usepackage{amssymb}
\usepackage{tikz}
\usepackage{quantikz}
\usepackage{braket}
\usepackage{blochsphere}
\usepackage{mdframed}
\geometry{margin=1.2in}

\usepackage{tcolorbox}
\tcbuselibrary{theorems}
\usepackage{xcolor}

\newtcbtheorem[number within=section]{mytheo}{Teorema}%
{colback=green!5,colframe=green!35!black,fonttitle=\bfseries}{th}

\newtcbtheorem[number within=section]{mydef}{Definición}%
{colback=blue!5,colframe=blue!35!black,fonttitle=\bfseries}{def}

\newtcbtheorem[number within=section]{mylem}{Lema}%
{colback=gray!5,colframe=gray!35!black,fonttitle=\bfseries}{lem}

\newtcbtheorem[number within=section]{mycor}{Corolario}%
{colback=orange!5,colframe=orange!35!black,fonttitle=\bfseries}{cor}


\titleformat{\section}
{\normalfont\Large\bfseries}{\thesection}{1em}{} % Corregido aquí

\title{Aritmética Utilizando la Transformada Cuántica de Fourier }
\author{Erick Jesús Ríos González}
\date{\today}

\begin{document}

\maketitle
\begin{abstract}
    Este es el resumen del documento. Aquí puedes escribir un breve resumen de tu trabajo, destacando los objetivos, métodos y conclusiones principales.
\end{abstract}
%%%%%%%%%%%%%%%%%%%%%%%%%%%%%%%%%%%%%%%%%%%%%%%%%%%%%%%%%%%%%%%%%%%%%%%%%%%%%%%%%%%%%
%%%%%%%%%%%%%%%%%%%%%%%%%%%%%%% INTRODUCCIÓN %%%%%%%%%%%%%%%%%%%%%%%%%%%%%%%%%%%%%%%%
%%%%%%%%%%%%%%%%%%%%%%%%%%%%%%%%%%%%%%%%%%%%%%%%%%%%%%%%%%%%%%%%%%%%%%%%%%%%%%%%%%%%%
\section{Introducción}
La transformada cuántica de Fourier nos permite cambiar de una 
base computacional a la base de Fourier. Nos permite pasar de 
la base usual (base computacional):
\[\ket{0} \& \ket{1}\]
\[\begin{bmatrix}
    1 \\
    0
\end{bmatrix} \&
\begin{bmatrix}
    0\\
    1
\end{bmatrix}\]
a la base:
\[\ket{+} \& \ket{-}\]
\[\frac{1}{\sqrt{2}}\begin{bmatrix}
    1 \\
    1
\end{bmatrix} \&\frac{1}{\sqrt{2}}
\begin{bmatrix}
    1\\
    -1
\end{bmatrix}\]
Es decir, tomando un ejemplo
podemos pasar de la siguiente representación de tres qubits:
\begin{figure}[h]
    \centering
    
    \begin{minipage}{0.3\textwidth}
        \centering
        \begin{blochsphere}[radius=1.5cm, tilt=15, rotation=-20, opacity=0.3, color=yellow]
            \labelLatLon{up}{90}{0};
            \labelLatLon{down}{-90}{90};
            \node[above] at (up) {{\tiny $\left|0\right>$ }};
            \node[below] at (down) {{\tiny $\left|1\right>$}};
            \drawBallGrid[style={opacity=0.1}]{30}{60}
            
            % Qubit 1: Estado |0>
            \drawStatePolar[axisarrow=true, color=green]{90}{0}{1}{\textcolor{blue}{$\ket{0}$}}
        \end{blochsphere}
        \caption*{Qubit 1: $|0\rangle$}
    \end{minipage}%
    \begin{minipage}{0.3\textwidth}
        \centering
        \begin{blochsphere}[radius=1.5cm, tilt=15, rotation=-20, opacity=0.3, color=orange]
            \labelLatLon{up}{90}{0};
            \labelLatLon{down}{-90}{90};
            \node[above] at (up) {{\tiny $\left|0\right>$ }};
            \node[below] at (down) {{\tiny $\left|1\right>$}};
            \drawBallGrid[style={opacity=0.1}]{30}{60}

            
            % Qubit 2: Estado |1>
            \drawStatePolar[axisarrow=true, color=green]{90}{180}{0}{\textcolor{blue}{$\ket{1}$}}
        \end{blochsphere}
        \caption*{Qubit 2: $|1\rangle$}
    \end{minipage}%
    \begin{minipage}{0.3\textwidth}
        \centering
        \begin{blochsphere}[radius=1.5cm, tilt=15, rotation=-20, opacity=0.3, color=red]
            \labelLatLon{up}{90}{0};
            \labelLatLon{down}{-90}{90};
            \node[above] at (up) {{\tiny $\left|0\right>$ }};
            \node[below] at (down) {{\tiny $\left|1\right>$}};
            \drawBallGrid[style={opacity=0.1}]{30}{60}

            
            % Qubit 3: Estado |1>
            \drawStatePolar[axisarrow=true, color=green]{90}{180}{0}{\textcolor{blue}{$\ket{1}$}}
        \end{blochsphere}
        \caption*{Qubit 3: $|1\rangle$}
    \end{minipage}

    \caption{La esfera de Bloch es una forma de representación gráfica del estado de 
    un qubit. En la parte superior de la esfera colocamos el estado \(\ket{0}\), 
    mientras que en la parte inferior el estado \(\ket{1}\). En el resto de la esfera
    colocamos todos los posibles estados en superposición. }
    \label{fig:blochsphere_tensor_professional}
\end{figure}
\begin{figure}[h]
    \centering
    
    \begin{minipage}{0.3\textwidth}
        \centering
        \begin{blochsphere}[radius=1.5cm, tilt=15, rotation=-20, opacity=0.3, color=yellow]
            \labelLatLon{up}{90}{0};
            \labelLatLon{down}{-90}{90};
            \node[above] at (up) {{\tiny $\left|0\right>$ }};
            \node[below] at (down) {{\tiny $\left|1\right>$}};
            \drawBallGrid[style={opacity=0.1}]{30}{60}
            
            % Qubit 1: Estado |0>
            \drawStatePolar[axisarrow=true, color=green]{0}{90}{270}{\textcolor{blue}{$\ket{0}$}}
        \end{blochsphere}
        \caption*{Qubit 1: $|0\rangle$}
    \end{minipage}%
    \begin{minipage}{0.3\textwidth}
        \centering
        \begin{blochsphere}[radius=1.5cm, tilt=15, rotation=-20, opacity=0.3, color=orange]
            \labelLatLon{up}{90}{0};
            \labelLatLon{down}{-90}{90};
            \node[above] at (up) {{\tiny $\left|0\right>$ }};
            \node[below] at (down) {{\tiny $\left|1\right>$}};
            \drawBallGrid[style={opacity=0.1}]{30}{60}

            
            % Qubit 2: Estado |1>
            \drawStatePolar[axisarrow=true, color=green]{0}{90}{180}{\textcolor{blue}{$\ket{1}$}}
        \end{blochsphere}
        \caption*{Qubit 2: $|1\rangle$}
    \end{minipage}%
    \begin{minipage}{0.3\textwidth}
        \centering
        \begin{blochsphere}[radius=1.5cm, tilt=15, rotation=-20, opacity=0.3, color=red]
            \labelLatLon{up}{90}{0};
            \labelLatLon{down}{-90}{90};
            \node[above] at (up) {{\tiny $\left|0\right>$ }};
            \node[below] at (down) {{\tiny $\left|1\right>$}};
            \drawBallGrid[style={opacity=0.1}]{30}{60}

            
            % Qubit 3: Estado |1>
            \drawStatePolar[axisarrow=true, color=green]{0}{90}{275}{\textcolor{blue}{$\ket{1}$}}
        \end{blochsphere}
        \caption*{Qubit 3: $|1\rangle$}
    \end{minipage}

    \caption{La esfera de Bloch es una forma de representación gráfica del estado de 
    un qubit. En la parte superior de la esfera colocamos el estado \(\ket{0}\), 
    mientras que en la parte inferior el estado \(\ket{1}\). En el resto de la esfera
    colocamos todos los posibles estados en superposición. }
    \label{fig:blochsphere_tensor_professional}
\end{figure}
%%%%%%%%%%%%%%%%%%%%%%%%%%%%%%%%%%%%%%%%%%%%%%%%%%%%%%%%%%%%%%%%%%%%%%%%%%%
%%%%%%%%%%%%%%%%%%% BINARY ENCODE Y %%%%%%%%%%%%%%%%%%%%%%%%%%%%%%%%%%%%%%%
%%%%%%%%%%%%%%%%%%%%%%%%%%%%%%%%%%%%%%%%%%%%%%%%%%%%%%%%%%%%%%%%%%%%%%%%%%%
\section{Representación Binaria de un Estado}
\begin{mydef}{Representación Binaria}{}
La representación binaria es un sistema de numeración en el que los números se expresan como combinaciones de potencias de 2, utilizando únicamente los dígitos 0 y 1. Cada dígito en una representación binaria se llama un bit. Por ejemplo, un número binario de \( n \) bits se puede expresar como:
\[
 \sum_{i=0}^{n-1} b_i \cdot 2^i = b_{n-1}b_{n-2}\ldots b_1b_0 
\]
donde \( b_i \) es el \( i \)-ésimo bit, con \( i \) variando desde 0 hasta \( n-1 \) de derecha a izquierda.
\end{mydef}
Sea \(y\in \mathbb{N}\cup\{0\}\). Utilizando la \texttt{Definición 3.1} proponemos la representación binaria
de \(y\) como: 
\[y= y_{n}y_{n-1}\ldots y_{0}\]
Para un sistema de qubits esta representación sería:
\[y= \ket{y_{n}y_{n-1}\ldots y_{0}}\]
\begin{mdframed}[linewidth=1pt, linecolor=orange]
    Ejemplo:

    \noindent Sea \(y= 3\), la representación binaria de \(y\) utilizando un sistema de qubits sería:
    \[\ket{3} = \ket{011}\]
    Es decir, con un sistema de tres qubits se pueden representar los siguientes estados:
    \[\{\ket{0},\ket{1},\ket{2},\ket{3},\ket{4},\ket{5},\ket{6},\ket{7}\}\]
    \[\{\ket{000},\ket{001},\ket{010},\ket{011},\ket{100},\ket{101},\ket{110},\ket{111}\}\]
    Recordemos esto, pues es lo que necesitamos para poder hacer sumas modulares utilizando la
    Transformada Cuántica de Fourier.
\end{mdframed}

Tambien necesitamos introducir notación para fracciones binarias,
lo cual nos ayudará a reescribir la Transformada Cuántica de Fourier (QFT)
de manera simplista.
\begin{mydef}{}{}
    Para $a_1, \ldots, a_m \in \{0, 1\}$ definimos
\[
0.a_1a_2 \ldots a_m := \frac{a_1}{2} + \frac{a_2}{4} + \ldots + \frac{a_m}{2^m} =
\sum_{l=1}^{m} a_l \cdot 2^{-l}.
\]
\end{mydef}
%%%%%%%%%%%%%%%%%%%%%%%%%%%%%%%%%%%%%%%%%%%%%%%%%%%%%%%%%%%%%%%%%%%%%%%%%%%
%%%%%%%%%%%%%%%%%%%%%%% SECTION QFT %%%%%%%%%%%%%%%%%%%%%%%%%%%%%
%%%%%%%%%%%%%%%%%%%%%%%%%%%%%%%%%%%%%%%%%%%%%%%%%%%%%%%%%%%%%%%%%%%%%%%%%%%
\section{La Transformada Cuántica de Fourier (QFT)}

\begin{mydef}{Transformada Integral}{}
    Una transformada integral es una operación lineal que convierte una función, \(f(x)\), en otra función, \(F(u)\), a través de la siguiente integral:
\[ F(u) = \int_{a}^{b} f(x) K(x,u) \, dx \]
La función \(K(x,u)\), conocida como el núcleo de la transformada, y los límites de la integral se especifican para una transformada particular.
\end{mydef}
\noindent El cambio de la base computacional a una base de Fourier se puede describir
como una transformada integral:
\[\{x\}\to\{y\}\]
\[IT[x] = ker(x,y)\{y\}\]
O en notación de nuestros vectores en el espacio de Hilbert:
\[QFT\ket{x} = ker(x,y)\ket{y}\]
Especificamente el núcleo que es de nuestro interés para esta transformada lo podemos
denotar como:
\[QFT\ket{x}=\frac{1}{\sqrt{N}}\sum_{y=0}^{N-1}e^{2\pi i \frac{xy}{N}}\ket{y}\]
Con lo que hemos obtenido nuestra primera definición de la Transformada Cuántica de Fourier:
\begin{mydef}{Transformada Cuántica de Fourier}{}
    Definimos la Transformada Cuántica de Fourier como:
    \[\frac{1}{\sqrt{N}} \sum_{i=0}^{N-1} e^{\frac{2\pi i x y}{y}} \left| k_i \right\rangle.\]
\end{mydef}

De esta manera, podemos escribir QFT para cualquier vector \(\ket{x}\) utilizando el 
siguiente lema:
\begin{mylem}{}{}
    Sea $n \in \mathbb{N}$ y
    \[
    x =\sum_{j=0}^{n-1} x_j 2^j, \quad \text{donde } x_j \in \{0, 1\} \text{ para } j \in \{0, \ldots, n-1\}.
    \]
    Entonces la acción de la transformada cuántica de Fourier $F$ sobre cualquier vector $|x\rangle$ de la base computacional de $\mathcal{H}_n$ puede escribirse como
    \[
    QFT\ket{x} = \frac{1}{\sqrt{2^n}} \bigotimes_{j=0}^{n-1} \left( \ket{0} + e^{2\pi i 0.x_j \ldots x_0} \ket{1} \right).
    \]
\end{mylem}



%%%%%%%%%%%%%%%%%%%%%%%%%%%%%%%%%%%%%%%%%%%%%%%%%%%%%%%%%%%%%%%%%%%%%%%%%%
%%%%%%%%%%%%%%%%%% EJEMPLO 1 + 2 %%%%%%%%%%%%%%%%%%%%%%%%%%%%%%%%%%%%%%%%%
%%%%%%%%%%%%%%%%%%%%%%%%%%%%%%%%%%%%%%%%%%%%%%%%%%%%%%%%%%%%%%%%%%%%%%%%%%

\section{Ejemplo 1+2}
Primero comenzamos por hacer la representación binaria de los números
\(A = 1 \text{ y } B=2 \), pero como el número \(A\) va a ser el que cargué la suma
tenemos que agregarle un qbit más para evitar los límites de la suma modular.
\[A = 1 = 001_2\]
\[B=2=10_2\]
Sea \(A=a_{2}a_{1}a_{0}\) y \(B=b_1b_0\) las representaciones binarias de \(A \text{ y } B\)
respectivamente, usando el teorema de reperesntación de Riez, podemos hacer la siguiente representación
para nuestro ejemplo:
\[\ket{001_2}= \ket{0}\otimes\ket{0}\otimes\ket{1}\]
 \[\ket{10_2}=\ket{1}\otimes\ket{0}\]
Para nuestro ejemplo seguiremos el siguiente circuito:

\begin{figure}[H]
    \centering
    \begin{quantikz}
        \lstick{$\ket{b_1}$} & \qw & \qw & \qw & \qw & \qw&\ctrl{3}&\qw&\ctrl{2}&\qw &\qw &\qw&\rstick{$\ket{b_1}$}\\
        \lstick{$\ket{b_0}$} & \qw & \qw & \qw & \ctrl{3} & \qw&\ctrl{}&\qw&\ctrl{}&\qw&\qw&\qw &\rstick{$\ket{b_0}$}\\
        \lstick{$\ket{a_2}$} & \qw & \qw & \gate[3]{QFT} & \qw & \qw &\qw&\qw&\gate{Z_3^{1}}&\qw&\gate[3]{QFT^{\dagger}}&\qw&\rstick{$\ket{(a+b)_2}$}\\
        \lstick{$\ket{a_1}$} & \qw & \qw & \qw & \qw & \qw&\gate{Z_2^{2}}&\qw&\qw&\qw&\qw&\qw&\rstick{$\ket{(a+b)_1}$}\\
        \lstick{$\ket{a_0}$} & \qw & \qw & \qw & \gate{Z_1^2} & \qw & \qw&\qw&\qw&\qw&\qw&\qw&\rstick{$\ket{(a+b)_0}$}\\
    \end{quantikz}
    \caption{Circuito cuántico para sumar dos números binarios}
    \label{fig:sumacuantica}
\end{figure}

Ahora que ya sabemos como vamos a operar usaremos el \texttt{Lema 1.1} y la \texttt{Deficinición 1.1} para aplicar la
QFT de nuestro número A:
\[QFT\ket{A} = QFT\ket{001}=\frac{1}{\sqrt{8}}\bigotimes_{j=0}^{2}\left(\ket{0}+\exp[2\pi i 0.a_j\ldots a_0]\ket{1}\right)\]
\[=\frac{1}{\sqrt{8}}\left[\left(\ket{0}+\exp[2\pi i 0.a_0]\ket{1}\right)\otimes \left(\ket{0}+\exp[2\pi i 0.a_1a_0]\ket{1}\right)\otimes \left(\ket{0}+\exp[2\pi i 0.a_2a_1a_0]\ket{1}\right)\right]\]
\[=\frac{1}{\sqrt{2}}\left(\ket{0}+\exp[2\pi i 0.1]\ket{1}\right)\otimes\frac{1}{\sqrt{2}}\left(\ket{0}+\exp[2\pi i 0.01]\ket{1}\right)\otimes\frac{1}{\sqrt{2}}\left(\ket{0}+\exp[2\pi i 0.001]\ket{1}\right)\]
\[=\frac{1}{\sqrt{2}}\left(\ket{0}+\exp[2\pi i\frac{a}{2}]\ket{1}\right)\otimes\frac{1}{\sqrt{2}}\left(\ket{0}+\exp[2\pi i \frac{a}{2^2}]\ket{1}\right)\otimes\frac{1}{\sqrt{2}}\left(\ket{0}+\exp[2\pi i \frac{a}{2^3}]\ket{1}\right)\]
\begin{equation*}
    = \underbrace{\frac{1}{\sqrt{2}}\left[\left(\ket{0}+\exp[2\pi i 0.1]\ket{1}\right)\right]}_{{\ket{\phi(a_2)}}} \otimes \underbrace{\frac{1}{\sqrt{2}}\left[\left(\ket{0}+\exp[2\pi i 0.01]\ket{1}\right)\right]}_{\ket{\phi(a_1)}} \otimes \underbrace{\frac{1}{\sqrt{2}}\left[\left(\ket{0}+\exp[2\pi i 0.001]\ket{1}\right)\right]}_{\ket{\phi(a_0)}}
\end{equation*}
Despues de la aplicación de QFT a el número A ahora tenemos que nuestro circuito se ha modificado de la siguiente forma:
\begin{figure}[H]
    \centering
    \begin{quantikz}
        \lstick{$\ket{b_1}$}        & \ctrl{4}       & \qw  & \ctrl{3}      & \qw & \qw           & \qw & \qw                    & \qw  & \qw & \rstick{$\ket{b_1}$} \\
        \lstick{$\ket{b_0}$}        & \ctrl{1}       & \qw  & \ctrl{1}      & \qw & \ctrl{1}      & \qw & \qw                    & \qw  & \qw & \rstick{$\ket{b_0}$} \\
        \lstick{$\ket{\phi(a_2)}$}  & \qw            & \qw  & \qw           & \qw & \gate{Z}      & \qw & \gate[3]{QFT^\dagger}  & \qw  & \qw & \rstick{$\ket{(a+b)_2}$} \\
        \lstick{$\ket{\phi(a_1)}$}  & \qw            & \qw  & \gate{Z}      & \qw & \qw           & \qw & \qw                    & \qw  & \qw & \rstick{$\ket{(a+b)_1}$} \\
        \lstick{$\ket{\phi(a_0)}$}  & \gate{Z}       & \qw  & \qw           & \qw & \qw           & \qw & \qw                    & \qw  & \qw & \rstick{$\ket{(a+b)_0}$}
    \end{quantikz}
    \caption{Circuito cuántico para sumar dos números binarios, después de haber aplicado QFT}
    \label{fig:sumacuantica2}
\end{figure}

Ahora procedemos a aplicar la compuerta Z tomando como control los qbits descritos en nuestro diagrama. Comenzando por el qbit \(\ket{\phi(a_0)}\) (qbit objetivo)
y \(\ket{b_0}\) (qbit control) aplicamos Z:
\[Z_3^2\ket{\phi(a_0)} = \frac{1}{\sqrt{2}}Z_3^2\ket{\left(\ket{0}+\exp[2\pi i 0.001]\ket{1}\right)}\]
\[=\frac{1}{\sqrt{2}}\left(\ket{0}+ \left(\exp[2\pi i 0.001]\exp[2\pi i 0.010]\right)\ket{1}\right)\]
\[=\frac{1}{\sqrt{2}}\left(\ket{0}+ \left(\exp[2\pi i (0.001+ 0.010)]\right)\ket{1}\right)\]
\[=\frac{1}{\sqrt{2}}\left(\ket{0}+ \left(\exp[2\pi i (0.011)]\right)\ket{1}\right)\]
Con lo cual hemos obtenido un nuevo elemento, el cual denotamos \(\ket{\phi(a_0)''}\):
\[\underbrace{\frac{1}{\sqrt{2}}\ket{\left(\ket{0}+ \exp[2\pi i (0.011)]\ket{1}\right)}}_{\ket{\phi(a_0)''}}\]
Haciendo un procedimiento análogo con los otros dos qbits, tenemos que para \(\ket{\phi(a_1)}\):
\[Z_2^2\ket{\phi(a_1)} = \frac{1}{\sqrt{2}}Z_2^2\ket{\left(\ket{0}+\exp[2\pi i 0.01]\ket{1}\right)}\]
\[=\frac{1}{\sqrt{2}}\left(\ket{0}+ \left(\exp[2\pi i (0.01+ 0.10)]\right)\ket{1}\right)\]
\[=\frac{1}{\sqrt{2}}\left(\ket{0}+ \left(\exp[2\pi i (0.11)]\right)\ket{1}\right)\]
Con lo cual hemos obtenido un nuevo elemento, el cual denotamos \(\ket{\phi(a_1)''}\)
\[\underbrace{\frac{1}{\sqrt{2}}\ket{\left(\ket{0}+ \exp[2\pi i (0.11)]\ket{1}\right)}}_{\ket{\phi(a_1)''}}\]
Finalmente para \(\ket{a_2}\) la rotación nos queda:
\[Z_1^1\ket{\phi(a_2)} = \frac{1}{\sqrt{2}}Z_1^1\ket{\left(\ket{0}+\exp[2\pi i 0.0]\ket{1}\right)}\]
\[=\frac{1}{\sqrt{2}}\left(\ket{0}+ \left(\exp[2\pi i (0.1+ 0.0)]\right)\ket{1}\right)\]
\[=\frac{1}{\sqrt{2}}\left(\ket{0}+ \left(\exp[2\pi i (0.1)]\right)\ket{1}\right)\]
Con lo cual hemos obtenido un nuevo elemento, el cual denotamos \(\ket{\phi(a_2)'}\)
\[=\underbrace{\frac{1}{\sqrt{2}}\ket{\left(\ket{0}+ \exp[2\pi i (0.1)]\ket{1}\right)}}_{\ket{\phi(a_2)'}}\]
Vemos que ahora nuestro diagrama se ha modificado a:
\begin{figure}[H]
    \centering
    \begin{quantikz}
        \lstick{$\ket{b_1}$}          &\qw&\qw &\qw&\qw&\rstick{$\ket{b_1}$}\\
        \lstick{$\ket{b_0}$}          &\qw&\qw&\qw &\qw&\rstick{$\ket{b_0}$}\\
        \lstick{$\ket{\phi(a_2)'}$}   &\qw&\qw&\gate[3]{QFT^{\dagger}}&\qw&\rstick{$\ket{(a+b)_2}$}\\
        \lstick{$\ket{\phi(a_1)''}$}  &\qw&\qw&\qw&\qw&\rstick{$\ket{(a+b)_1}$}\\
        \lstick{$\ket{\phi(a_0)''}$}  &\qw&\qw&\qw&\qw&\rstick{$\ket{(a+b)_0}$}\\
    \end{quantikz}
    \caption{Circuito cuántico para sumar dos números binarios, después de haber aplicado QFT, recordemos que un último paso de \(QFT^{\dagger}\)
    es hacer un swap entre \(\) y las rotaciones correspondientes}
    \label{fig:sumacuantica3}
\end{figure}

Finalmente aplicamos \(QFT^{\dagger}\) al ket conformado por \(\ket{\phi(a_2)'\phi(a_1)''\phi(a_0)''}\), esta operación se puede ver como:
\[QFT^{\dagger}\ket{\phi(a_2)'\otimes\phi(a_1)''\otimes\phi(a_0)''}=QFT^{\dagger}\ket{\phi(a_2)'}\otimes QFT^{\dagger}\ket{\phi(a_1)''}\otimes QFT^{\dagger}\ket{\phi(a_0)''}\]
Recordemos que \(QFT^{\dagger}\) realiza un swap, para este ejemplo, entre $\ket{\phi(a_2)'}$ y $\ket{\phi(a_1)''}$. Para de esta manera obtener
la componente \(\ket{(a+b)_2}\text{, }\ket{(a+b)_1} \text{ y } \ket{(a+b)_0}\).
\[=\ket{\ket{0}\otimes\ket{1}\otimes\ket{1}}\]
\[=\ket{011}\]
\[\implies 011_2 = 3_{10}=1+2\]

%%%%%%%%%%%%%%%%%%%% ends document %%%%%%%%%%%%%%%%%%%%%%%%%%%5%%%%
\end{document}