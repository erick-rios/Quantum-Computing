\documentclass{article}
\usepackage[utf8]{inputenc}
\usepackage{graphicx} % Required for inserting images
\usepackage{geometry}
\usepackage{pgfplots}
\usepackage{float}
\usepackage{enumitem}
\usepackage{titlesec}
\usepackage{amssymb}


\geometry{margin=1.2in}

\usepackage[style=apa, backend=biber]{biblatex}
\usepackage{tcolorbox}
\tcbuselibrary{theorems}
\usepackage{xcolor}

\newtcbtheorem[number within=section]{mytheo}{Theorem}%
{colback=green!5,colframe=green!35!black,fonttitle=\bfseries}{th}

\newtcbtheorem[number within=section]{mydef}{Definition}%
{colback=blue!5,colframe=blue!35!black,fonttitle=\bfseries}{def}



\titleformat{\section}
{\normalfont\Large\bfseries}{\thesection}{1em}{} % Corregido aquí

\title{Introduction to Mathematics of Quantum Computing}
\author{Erick Jesús Ríos González}
\date{\today}

\begin{document}

\maketitle
\noindent Quantum mechanics describes states of systems
at microscopic scales, which generally means that only
speaking in terms of probabilities is allowed.
The state in a quantum system can be described as an
element of a vector space.
It is possible that the state of the system can be written as
a linear combination of a basis of the vector space: \textcolor{red}{superposition}.
The observables of a quantum system are those physical quantities that
in principle can be determined simultaneously and with a precision constraint:
\textcolor{red}{Heisenberg's uncertainty principle}. In other words, they can be measurable.

Measurements on objects prepared "in the same way" are
distributed with a relative frequency around a mean value.

\begin{equation*}
    \text{relative frequency} = \dfrac{\text{Number of measurements with result }a}{\text{Total number of measurements}}
\end{equation*}
\begin{equation*}
    \text{mean value} = \sum_{a\in Measurements} a\times (\text{relative frequency of measurements with result } a)
\end{equation*}
\section{Hilbert Spaces}
There are initial preparations whose state can be characterized with a vector
in a Hilbert space (vector space).
\begin{enumerate}
    \item States that can be characterized with a vector are called
    pure states.
    \item Otherwise, we speak of "mixed states".
\end{enumerate}
\begin{mydef}{Hlbert Space}{defexample}
    A Hilbert Space $\mathbb{H}$ is a complete complex vector space, that is,
    \begin{equation}
        \Psi,\Phi \in \mathbb{H} \text{ and } a,b \in \mathbb{C} \implies a\Psi+b\Phi \in \mathbb{H}
    \end{equation}
    with a positive-definite scalar product:
    \begin{equation}
        \langle\cdot|\cdot\rangle: \mathbb{H}\times\mathbb{H}\to \mathbb{C}
    \end{equation}
    such that $\forall \Psi,\Phi, \Psi_1, \Psi_2 \in \mathbb{H}$ and $a,b \in \mathbb{C}$
    \begin{equation*}
        \langle\Psi|\Phi\rangle = {\langle\Psi|\Phi\rangle}
    \end{equation*}
\end{mydef}
\end{document}