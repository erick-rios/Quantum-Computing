\documentclass{article}
\usepackage[utf8]{inputenc}
\usepackage{graphicx} % Required for inserting images
\usepackage{geometry}
\usepackage{pgfplots}
\usepackage{float}
\usepackage{enumitem}
\usepackage{titlesec}
\usepackage{amssymb}


\geometry{margin=1.2in}

\usepackage[style=apa, backend=biber]{biblatex}
\usepackage{graphicx}
\usepackage{float}
\usepackage{geometry}
\usepackage{tcolorbox}
\tcbuselibrary{theorems}
\usepackage{xcolor}

\newtcbtheorem[number within=section]{mytheo}{Theorem}%
{colback=green!5,colframe=green!35!black,fonttitle=\bfseries}{th}

\newtcbtheorem[number within=section]{mydef}{Definition}%
{colback=blue!5,colframe=blue!35!black,fonttitle=\bfseries}{th}



\titleformat{\section}
{\normalfont\Large\bfseries}{\thesection}{1em}{}

\title{Introduction to Mathematics of Quantum Computing}
\author{Erick Jesús Ríos González}
\date{\today}

\begin{document}

\maketitle
\noindent A quantum computer is a device that performs computational operations 
taking advantage of specific properties described by quantum mechanics.
%%%%%%%%%%%%%%%%% begins the document %%%%%%%%%%%%%%%%
\begin{mytheo}{This is my title}{theoexample}
    This is the text of the theorem. The counter is automatically assigned and,
    in this example, prefixed with the section number. This theorem is numbered with
    \ref{th:theoexample} and is given on page \pageref{th:theoexample}.
\end{mytheo}
\begin{mydef}{Hlbert Space}{defexample}
    A Hilbert Space $\mathbb{H}$ is a complete complex vector space, that is,
    \begin{equation}
        \Psi,\Phi \in \mathbb{H} \text{ and } a,b \in \mathbb{C} \implies a\Psi+b\Phi \in \mathbb{H}
    \end{equation}
    with a positive-definite scalar product:
    \begin{equation}
        \langle\cdot|\cdot\rangle: \mathbb{H}\times\mathbb{H}\to \mathbb{C}
    \end{equation}
    such that $\forall \Psi,\Phi, \Psi_1, \Psi_2 \in \mathbb{H}$ and $a,b \in \mathbb{C}$
    \begin{equation*}
        \langle\Psi|\Phi\rangle = {\langle\Psi|\Phi\rangle}
    \end{equation*}
    \begin{enumerate}
        \item Primero definimos el invariante de ciclo. De nuestras clases de análisis dos podemos reordenar la serie de este problema bajo la siguiente definición:
    \begin{mydef}{Convergencia Absoluta}{}
        Se dice que la serie $\sum a_{n}$ es absolutamente convergente si la serie $\sum\left\vert a_{n}\right\vert$ es convergente . 
    
        $\sum a_{n}$ es absolutamente convergente $ \Longrightarrow \sum a_{n}$ es convergente .
    \end{mydef}
    esto nos permite escribir el invariante de ciclo como:
    \begin{equation*}
        total = \sum_{j=0}^{k-1-{i+1}}A[j+i+1]\cdot x_0^j
    \end{equation*}
    ya que se trata de una serie de elementos finitos que es absolutamente convergente.
    Al inicio del ciclo, cuando $i=k-1$ y $\text{total} = 0$ por la suposición podemos demostrar que:
    \begin{equation*}
            \text{total} = \sum_{j=0}^{k-1-(k-1+1)} A[k-(k-1)+1]\cdot x_0^{(k-1)}
    \end{equation*}
    \begin{equation*}
        \sum_{j=0}^{-1} A[k-(k-1)+1]\cdot x_0^{(k-1)}=0
    \end{equation*}
    Como la suma es vacía, el invariante de ciclo es correcto incluso antes de inicializar el ciclo.
    \item Caso base (i=0):
    \begin{equation*}
        \text{total} = \sum_{j=0}^{-(k-1+1)} A[k-(k-1)+1]\cdot x_0^{(k-1)}
    \end{equation*}
    \item Hipótesis de inducción:
\end{mydef}
\end{document}